\documentclass[10pt]{beamer}

\usetheme{Warsaw}
\beamertemplatenavigationsymbolsempty

\usepackage[utf8x]{inputenc}
\usepackage[francais]{babel}
\usepackage{hyperref}
\usepackage{amsmath}
\usepackage{graphicx}
\usepackage{tikz}
\usepackage{multicol}
\usetikzlibrary{automata,positioning}
\graphicspath{{./img/}}
\DeclareGraphicsExtensions{.png, .jpeg, .jpg}


\renewcommand*\thesection{\arabic{section}}


\AtBeginSection[]{%
  \begin{frame}<beamer>
    \frametitle{Plan}
    \tableofcontents[sectionstyle=show/hide,subsectionstyle=hide/show/hide]
  \end{frame}
  \addtocounter{framenumber}{-1}
}

\setbeamertemplate{footline}[frame number]


\title{Statistical Machine Translation for Query Expansion in Answer
  Retrieval\\ 
\small
Stefan Riezler, Alexander Vasserman, Ioannis Tsochantaridis, Vibhu
Mittal and Yi Liu}

\author{Rémi Bois, Agathe Mollé}
\date{\today}

\begin{document}

\begin{frame}
  \maketitle
  \vfill
  \begin{figure}
    \includegraphics[width=0.20\textwidth]{logo_univ_nantes}
  \end{figure}

\end{frame}

\begin{frame}
  \tableofcontents
\end{frame}

\section{Introduction}
\label{sec:intro}


\section{Question Answering}
\label{sec:QA}

\begin{frame}
  \frametitle{Répondre à des questions}
  % questions en langage naturel, réponses en langage naturel
\end{frame}

\begin{frame}
  \framtitle{Une tâche de Recherche d'Informations}
  % recherche dans un ensemble de FAQ
\end{frame}

\section{Ajout de la paraphrase et de la traduction de questions}
\label{sec:paratrans}

\begin{frame}
  \frametitle{Paraphraser pour améliorer la précision}
  % présenter l'intuition et un exemple
\end{frame}

\begin{frame}
  \frametitle{Comment paraphraser ?}
  % double traduction
\end{frame}

\begin{frame}
  \frametitle{Sélection des paraphrases}
  %formule
\end{frame}

\begin{frame}
  \frametitle{La traduction automatique pour améliorer les résultats
    ?}
  % traduction question -> réponse + exemple
\end{frame}

\section{Corpus et données d'entraînement}
\label{sec:corpus}

\begin{frame}
  \frametitle{L'entraînement du module de paraphrase}
  % russe truc
\end{frame}

\begin{frame}
  \frametitle{Le corpus composé de FAQ}
  % présentation du corpus, stats etc...
\end{frame}


\section{Résultats}
\label{sec:results}


\begin{frame}
  \frametitle{Baseline}
  % description de la baseline
  % description du S_2@n
\end{frame}

\begin{frame}
  \frametitle{Résultats}
  % le tableau
\end{frame}

\begin{frame}
  \frametitle{Explication des résultats}
  % tableau 4 de l'article
\end{frame}

\section{Conclusion}
\label{sec:conclusion}


\begin{frame}
  \frametitle{Une méthode efficace}
  % approche nouvelle (?)
\end{frame}

\begin{frame}
  \frametitle{Quelques limites}
  % stemming ? taille du corpus de test ?
\end{frame}


\end{document}